% Options for packages loaded elsewhere
\PassOptionsToPackage{unicode}{hyperref}
\PassOptionsToPackage{hyphens}{url}
\PassOptionsToPackage{dvipsnames,svgnames,x11names}{xcolor}
%
\documentclass[
  letterpaper,
  DIV=11,
  numbers=noendperiod]{scrartcl}

\usepackage{amsmath,amssymb}
\usepackage{lmodern}
\usepackage{iftex}
\ifPDFTeX
  \usepackage[T1]{fontenc}
  \usepackage[utf8]{inputenc}
  \usepackage{textcomp} % provide euro and other symbols
\else % if luatex or xetex
  \usepackage{unicode-math}
  \defaultfontfeatures{Scale=MatchLowercase}
  \defaultfontfeatures[\rmfamily]{Ligatures=TeX,Scale=1}
\fi
% Use upquote if available, for straight quotes in verbatim environments
\IfFileExists{upquote.sty}{\usepackage{upquote}}{}
\IfFileExists{microtype.sty}{% use microtype if available
  \usepackage[]{microtype}
  \UseMicrotypeSet[protrusion]{basicmath} % disable protrusion for tt fonts
}{}
\makeatletter
\@ifundefined{KOMAClassName}{% if non-KOMA class
  \IfFileExists{parskip.sty}{%
    \usepackage{parskip}
  }{% else
    \setlength{\parindent}{0pt}
    \setlength{\parskip}{6pt plus 2pt minus 1pt}}
}{% if KOMA class
  \KOMAoptions{parskip=half}}
\makeatother
\usepackage{xcolor}
\setlength{\emergencystretch}{3em} % prevent overfull lines
\setcounter{secnumdepth}{-\maxdimen} % remove section numbering
% Make \paragraph and \subparagraph free-standing
\ifx\paragraph\undefined\else
  \let\oldparagraph\paragraph
  \renewcommand{\paragraph}[1]{\oldparagraph{#1}\mbox{}}
\fi
\ifx\subparagraph\undefined\else
  \let\oldsubparagraph\subparagraph
  \renewcommand{\subparagraph}[1]{\oldsubparagraph{#1}\mbox{}}
\fi


\providecommand{\tightlist}{%
  \setlength{\itemsep}{0pt}\setlength{\parskip}{0pt}}\usepackage{longtable,booktabs,array}
\usepackage{calc} % for calculating minipage widths
% Correct order of tables after \paragraph or \subparagraph
\usepackage{etoolbox}
\makeatletter
\patchcmd\longtable{\par}{\if@noskipsec\mbox{}\fi\par}{}{}
\makeatother
% Allow footnotes in longtable head/foot
\IfFileExists{footnotehyper.sty}{\usepackage{footnotehyper}}{\usepackage{footnote}}
\makesavenoteenv{longtable}
\usepackage{graphicx}
\makeatletter
\def\maxwidth{\ifdim\Gin@nat@width>\linewidth\linewidth\else\Gin@nat@width\fi}
\def\maxheight{\ifdim\Gin@nat@height>\textheight\textheight\else\Gin@nat@height\fi}
\makeatother
% Scale images if necessary, so that they will not overflow the page
% margins by default, and it is still possible to overwrite the defaults
% using explicit options in \includegraphics[width, height, ...]{}
\setkeys{Gin}{width=\maxwidth,height=\maxheight,keepaspectratio}
% Set default figure placement to htbp
\makeatletter
\def\fps@figure{htbp}
\makeatother

\KOMAoption{captions}{tableheading}
\makeatletter
\makeatother
\makeatletter
\makeatother
\makeatletter
\@ifpackageloaded{caption}{}{\usepackage{caption}}
\AtBeginDocument{%
\ifdefined\contentsname
  \renewcommand*\contentsname{Table of contents}
\else
  \newcommand\contentsname{Table of contents}
\fi
\ifdefined\listfigurename
  \renewcommand*\listfigurename{List of Figures}
\else
  \newcommand\listfigurename{List of Figures}
\fi
\ifdefined\listtablename
  \renewcommand*\listtablename{List of Tables}
\else
  \newcommand\listtablename{List of Tables}
\fi
\ifdefined\figurename
  \renewcommand*\figurename{Figure}
\else
  \newcommand\figurename{Figure}
\fi
\ifdefined\tablename
  \renewcommand*\tablename{Table}
\else
  \newcommand\tablename{Table}
\fi
}
\@ifpackageloaded{float}{}{\usepackage{float}}
\floatstyle{ruled}
\@ifundefined{c@chapter}{\newfloat{codelisting}{h}{lop}}{\newfloat{codelisting}{h}{lop}[chapter]}
\floatname{codelisting}{Listing}
\newcommand*\listoflistings{\listof{codelisting}{List of Listings}}
\makeatother
\makeatletter
\@ifpackageloaded{caption}{}{\usepackage{caption}}
\@ifpackageloaded{subcaption}{}{\usepackage{subcaption}}
\makeatother
\makeatletter
\@ifpackageloaded{tcolorbox}{}{\usepackage[many]{tcolorbox}}
\makeatother
\makeatletter
\@ifundefined{shadecolor}{\definecolor{shadecolor}{rgb}{.97, .97, .97}}
\makeatother
\makeatletter
\makeatother
\ifLuaTeX
  \usepackage{selnolig}  % disable illegal ligatures
\fi
\IfFileExists{bookmark.sty}{\usepackage{bookmark}}{\usepackage{hyperref}}
\IfFileExists{xurl.sty}{\usepackage{xurl}}{} % add URL line breaks if available
\urlstyle{same} % disable monospaced font for URLs
\hypersetup{
  pdftitle={Introduccion\_R\_Morocho\_Nathalia},
  pdfauthor={Morocho\_Nathalia},
  colorlinks=true,
  linkcolor={blue},
  filecolor={Maroon},
  citecolor={Blue},
  urlcolor={Blue},
  pdfcreator={LaTeX via pandoc}}

\title{Introduccion\_R\_Morocho\_Nathalia}
\author{Morocho\_Nathalia}
\date{}

\begin{document}
\maketitle
\ifdefined\Shaded\renewenvironment{Shaded}{\begin{tcolorbox}[frame hidden, breakable, interior hidden, boxrule=0pt, enhanced, sharp corners, borderline west={3pt}{0pt}{shadecolor}]}{\end{tcolorbox}}\fi

\#asignacion de valores a \textless- 6

\#creacion de vectores b \textless- c(6,5,8,9,12,18)

\#posicion dentro del vector b{[}2{]} b{[}5{]} b{[}1:3{]}

\#mostrar el vector, eliminando la posicion indicada b{[}-6{]}

\#tablas de datos d \textless- c(3,6,8) e \textless- c(4,7,6) f
\textless- c(4,5,6)

d \textless- data.frame(d,e,f)

\#graficar con la libreria ggplot2 library(ggplot2)

\#graficar un punto g \textless- 4 h \textless- 9

dat \textless- data.frame(g,h)

ggplot() + geom\_point(data=dat,aes(x=g,y=h),size=5,color=``blue'')

\#graficar y configurar los ejes, etiquetas i \textless- c(2,5,1) j
\textless- c(6,4,9)

dat1 \textless- data.frame(i,j) ggplot() + geom\_point(data = dat1,
aes(x=i,y=j),size=5,color=``green'') +
scale\_x\_continuous(limits=c(0,15),breaks = seq(0,15,1)) +
scale\_y\_continuous(limits=c(0,15),breaks = seq(0,15,1))

\#uso de colores y figuras en ggplot ggplot() + geom\_point(data = dat1,
aes(x=i,y=j),size=10,color=``brown'',shape=10) +
scale\_x\_continuous(limits=c(0,15),breaks = seq(0,15,1)) +
scale\_y\_continuous(limits=c(0,15),breaks = seq(0,15,1))

\#graficar lineas k \textless- c(1,8) l \textless- c(3,10)

dat2 \textless- data.frame(k,l)

ggplot() + geom\_line(data = dat2, aes(x=k,y=l))

\#cambiar datos dat2\(k <- c(1,8) dat2\)l \textless- c(3,10)

ggplot() + geom\_line(data = dat2, aes(x=k,y=l))

\#graficar lines con margen de datos m \textless- c(0,10) n \textless-
3*x+1

dat3 \textless- data.frame(m,n)

ggplot() + geom\_line(data = dat3, aes(x=m,y=n))

n \textless- .5*x - .73

dat3 \textless- data.frame(m,n)

ggplot() + geom\_line(data = dat3, aes(x=m,y=n))

\#datos de lineas de tendencia sample (1:10,100,replace = TRUE)

rnorm(100,50,10) rnorm(100,50,90)

\#graficas con datos de tendencia rep (1,100)

o \textless- rep(1,100) p \textless- rnorm (100,50,10)

dat4 \textless- data.frame(o,p)

o \textless- 1 p \textless- 50

mean \textless- data.frame(o,p)

ggplot() + geom\_point(data = dat4, aes(x=o,y=p))+ geom\_point(data =
mean,aes(x=o,y=p),size=7,color=``purple'')

\#puntos criticos q \textless- rep(1,100) q \textless- c(x,rep(9,100)) r
\textless- rnorm(100,50,10) r \textless- c(r,rnorm(100,30,10)) r
\textless- c(r,rnorm(100,78,10))

dat5 \textless- data.frame(q,r)

q \textless- c(1,9,15) r \textless- c(50,30,78)

means \textless- data.frame (q,r)

ggplot() + geom\_point(data = dat5, aes(x=q,y=r))+ geom\_point(data =
means,aes(x=q,y=r),size=7,color=``red'') ---



\end{document}
